%% LyX 2.0.3 created this file.  For more info, see http://www.lyx.org/.
%% Do not edit unless you really know what you are doing.
\documentclass[british]{report}
\usepackage{mathptmx}
\usepackage{helvet}
\usepackage[T1]{fontenc}
\usepackage[latin9]{inputenc}
%\usepackage{geometry}
%\geometry{verbose,tmargin=2cm,bmargin=2cm,lmargin=3cm,rmargin=2cm}
\setcounter{secnumdepth}{3}
\setcounter{tocdepth}{3}
\usepackage{verbatim}
\usepackage{wrapfig}
\usepackage{textcomp}
\usepackage{graphicx}
\usepackage{setspace}
\usepackage{nomencl}
% the following is useful when we have the old nomencl.sty package
\providecommand{\printnomenclature}{\printglossary}
\providecommand{\makenomenclature}{\makeglossary}
\makenomenclature
\doublespacing

\makeatletter

%%%%%%%%%%%%%%%%%%%%%%%%%%%%%% LyX specific LaTeX commands.
\DeclareRobustCommand{\greektext}{%
  \fontencoding{LGR}\selectfont\def\encodingdefault{LGR}}
\DeclareRobustCommand{\textgreek}[1]{\leavevmode{%
  \IfFileExists{grtm10.tfm}{}{\fontfamily{cmr}}\greektext #1}}
\DeclareFontEncoding{LGR}{}{}
\DeclareTextSymbol{\~}{LGR}{126}

%%%%%%%%%%%%%%%%%%%%%%%%%%%%%% User specified LaTeX commands.
% include the comments when we output. This is so we get the bibliography when previewing this chapter on its own
\usepackage{comment}
\includecomment{comment}

% stop LaTeX from putting hyphens everywhere
\hyphenpenalty=10000

% Make figure captions look nice. Small, sans-serif, bold Fig X.X bit
\usepackage[margin=10pt,font={small,sf},labelfont=bf,labelsep=endash]{caption}

% Make chapter headings use a sans-serif font
\usepackage[{sf,bf}]{titlesec}

% centre tables and figures on the page
\let\originaltabular\tabular
\let\endoriginaltabular\endtabular
\renewenvironment{tabular}[1]{%
  \begingroup%
  \centering%
  \originaltabular{#1}}%
  {\endoriginaltabular\endgroup}

\let\originaltable\table
\let\endoriginaltable\endtable
\renewenvironment{table}[1][ht]{%
  \originaltable[#1]
  \centering}%
  {\endoriginaltable}

\let\originalfigure\figure
\let\endoriginalfigure\endfigure
\renewenvironment{figure}[1][ht]{%
  \originalfigure[#1]
  \centering}%
  {\endoriginalfigure}

\usepackage[landscape,a3paper,left=21cm,marginparsep=1cm]{geometry}
\marginparwidth=\textwidth
\reversemarginpar
\usepackage[style=numeric]{biblatex}
\addbibresource{/Users/phil/Dropbox/Work/BibTeX/THESIS}

\makeatother

\usepackage{babel}
\begin{document}

\chapter{Introduction}

The nucleus of a human cell contains approximately 2 metres of DNA
\cite{Annunziato2008}\marginpar{\fullcite{Annunziato2008}}, requiring a huge degree of compaction to fit
within the nucleus. In addition to storing DNA the nucleus itself
is a highly functional organelle, responsible for the huge variety
of complex cellular processes. To achieve this, the contents of the
nucleus are very precisely organised into sub-nuclear compartments
which specialise in processes such as transcription and replication.
In addition to the organisation of the nuclear proteome, chromatin
itself is thought to be organised. In this chapter I will discuss
the current understanding of nuclear structure and organisation, how
we have obtained this knowledge, and how it may play a part in the
initiation of certain cancers.


\section{Chromatin}

Chromatin is a general term used to describe the DNA found in the
nucleus along with the plethora of proteins that bind to it. There
are two principle forms of chromatin: \emph{euchromatin }and \emph{heterochromatin}.
These were first described in the early twentieth century by the difference
in staining within the nucleus \cite{Heitz1928}. 

Heterochromatin stains darkly because it remains highly condensed
during interphase, typically relocating to the nuclear periphery.
can be split into \emph{constitutive heterochromatin} and \emph{facultative
heterochromatin}; the former describing heterochromatin found in all
cell types typically containing repetitive elements found in centromeres
and telomeres; the latter describing less compacted inactive chromatin
which may vary between cell types as they differentiate. Heterochromatin
is well known as being a repressive environment for gene expression;
studies whereby normally active regions are artificially tethered
to the inner nuclear membrane show the ablation of gene expression
\cite{Finlan2008,Reddy2008} though this effect appears to be locus
specific and is not always observed \cite{Kumaran2008}.  Euchromatin
is the site of most geneic transcription, as shown by the presence
of RNA Polymerase and nascent transcriptions found in early fractionation
studies \cite{Chesterton1974}. Its looser compaction allows access
to the DNA by the cellular machinery; enabling the binding of transcription
factors and the initiation of transcription.

The differences between heterochromatin and euchromatin lie within
the proteins that they contain. Chromatin acts as a platform for proteins
to bind to, differences in histone modifications, variants, nucleosome
packing and DNA modifications affect the accessibility and binding
profile of the chromatin, and so how the DNA is interpreted by the
cell.


\subsection{Histones}

\begin{wrapfigure}{o}{0.4\columnwidth}%
\begin{centering}
\includegraphics[width=0.4\columnwidth]{\string"figures/chapter 1/Nucleosome_1KX5_colour_coded\string".png}
\par\end{centering}

\caption[Structure of the nucleosome]{\label{fig:nucleosome-xray-structure}\textbf{Structure of the nucleosome.
}X Ray Structure of a Nucleosome core particle at a resolution of
1.9�. DNA can be seen wrapped around the core histones. PDB structure
1KX5 \cite{Davey2002}.}
\end{wrapfigure}%
To package DNA, the double helix is wrapped around an octomer of core
histones: two H2A, two H2B, two H3 and two H4. 146 base pairs of DNA
wrap around these positively charged proteins to form the nucleosome,
which is then bound by histone H1 with linker DNA to make a total
of 166 base pairs \cite{Davey2002}. This packing forms the 10 nm
fibre, often known as 'beads on a string' due to its appearance in
electron micrographs. At their most basic level, histones function
to compact DNA by counteracting the negative charge of the phosphorylated
back bone. 


\subsection{Histone modifications}

Core histones have flexible tails which extend outside of the nucleosome,
and are accessible to proteins within the nuclear matrix. These tails
can be post-translationally modified at a large number of residues
and these modifications can affect the packing of chromatin \cite{Wolffe1999a}
as well as which proteins can bind. The large number of combinatorial
possibilities that result result from these modifications have been
dubbed the 'Histone Code' \cite{Strahl2000a}, different modifications
are related to different chromatin states: for example, active promoters
are typically enriched for di- and tri-methylation of histone 3 lysine
4 (H3K4), inactive promoters for H3K27me3 and H3K9me3 (Figure \ref{fig:Histone-mod-dashboard})
(for a review, see \cite{Zhou2011}).

As chromatin immunoprecipitation (\nomenclature{ChIP}{Chromatin immunoprecipitation}ChIP)
has become a common laboratory technique and been combined with microarray
techniques (ChIP on chip) and next generation sequencing (ChIP-Seq),
our understanding of how these marks affect chromatin biology on a
genome-wide scale has advanced dramatically. Profiling chromatin types
using multiple datasets covering a large number of histone modifications
is sufficient to predict the identity and function of regions in the
genome with a high degree of accuracy, revealing previously unknown
enhancers \cite{Heintzman2007,Ernst2010,Hon2009}. Ernst \emph{et
al.} used the genome-wide profiles of 9 histone modifications in 9
different cell types to define 15 different chromatin states, describing
different states of promoters, enhancers and insulators, amongst others\cite{Ernst2011}.
They integrated data from genome-wide association studies (\nomenclature{GWAS}{Genome-wide association study}GWAS)
and found numerous enhancer elements which coincide with disease associated
mutations. This systems biology approach can reveal the dynamics across
different cell types and can be a powerful tool in understanding how
the genome is interpreted in health and disease.

\begin{figure}
\includegraphics[width=0.8\columnwidth]{\string"figures/chapter 1/nrg2905-f4\string".jpg}

\caption[Histone modification associations in chromatin]{\label{fig:Histone-mod-dashboard}\textbf{Histone modification associations
in chromatin.} Diagram showing the range of histone modifications
available within mammalian chromatin and how they can specify the
behaviour of elements within the DNA. Taken from\cite{Zhou2011}.}
\end{figure}



\subsection{Histone variants}

In addition to histone tail modifications, chromatin can be modified
by the incorporation of histone variants. Canonical core histone genes
are found in clustered repeat arrays within the genome, are transcribed
during replication and are extremely tightly conserved between species.
Histone variants are found as single genes spread through the genome,
and are subject to far greater diversity \cite{Talbert2010}.

CENP-A is a human variant of histone H3 which replaces the canonical
histone in centromeric heterochromatin. It is a key factor in the
establishment of the centromeres and kinetochores required for mitosis.
The histone variant is incorporated with the help of a number of chaperone
proteins, including HJURP, after replication of DNA has finished \cite{Dunleavy2009,Foltz2009}.
CENP-A is essential for the formation of centromeres.

Another frequent histone variant found in humans is H3.3, which differs
from canonical H3 by just four amino acids \cite{Talbert2010}. This
histone variant is found within transcribed genes, promoters and regulatory
elements, and is thought to be laid down during transcriptional elongation
\cite{Schwartz2005}. Nucleosomes containing H3.3 appear to be less
stable than canonical nucleosomes, with a high turnover of H3.3 \cite{Schwartz2005}.
It is possible that this increased turnover of the nucleosomal components
helps to keep the chromatin open and accessible to the transcriptional
machinery \cite{Talbert2010}.

Other core histone proteins also have variants, such as H2A.Z, a histone
variant found on either side of the \emph{nucleosome free regions
}found at the transcriptional start sites of active genes as well
as insulator regions \cite{Zlatanova2008}. H2A.Z promotes the recruitment
of RNA Polymerase II through mediating protein interactions \cite{Adam2001}.





\section{Two-dimensional organisation}

Chromosome banding, \emph{cis} organisation of Hox genes, gene clusters,
gene rich areas and gene deserts. 


\section{Techniques to investigate nuclear organisation}


\subsection{Microscopy}


\subsubsection{Chromosome painting}


\subsubsection{DNA-FISH}


\subsubsection{RNA-FISH}


\subsubsection{Electron microscopy}


\subsection{Chromosome conformation capture}

Chromosome conformation capture (\nomenclature{3C}{Chromosome conformation capture}3C)


\subsubsection{Circular 4C}


\subsubsection{e4C}


\subsubsection{5C}


\subsubsection{ChIA-PET}


\subsubsection{Hi-C}


\subsubsection{Tethered Hi-C}


\section{Chromosome Territories}

The interphase nucleus is a highly structured organelle. As chromosomes
decondense after metaphase they retain some degree of structure, forming
\emph{chromosome territories} (\nomenclature{CT}{Chromosome territory}CTs)
\cite{Cremer2001}. Circumstantial evidence for interphase organisation
of chromosomes has existed for a long time, first suggested by Carl
Rabl in 1885 \cite{Rabl1885}. Oberservations by Stack \emph{et al.}
using microscopy with giemsa-band staining suggested that chromosomes
retained some degree of organisation during interphase \cite{Stack1977},
and in 1982 Cremer \emph{et al., }showed that interphase chromosomes
exist in 'territories' by studying the pattern of DNA damage in metaphase
chromosomes after spot irradiation in during interphase \cite{Cremer1982}.
The subsequent development of 'chromosome paints', a method to visualise
entire- or part- chromosomes with fluorescence \emph{in-situ} hybridisation
(\nomenclature{FISH}{Fluorescence in-situ hybridisation}FISH), confirmed
these findings \cite{Schardin1985a,Manuelidis1985}.

As FISH techniques have developed, so too has the detail with which
CT organisation can be studied. Several groups have shown that chromosome
territory position within the nucleus is not random and correlates
with chromosome size \cite{Sun2000,Cremer2001b}, gene-density \cite{Cremer2001b,Croft1999}
and replication timing \cite{Visser1998}: those near the centre of
the nucleus tend to be gene-rich, early replicating and small. The
specificity of CT positioning is a topic of debate, though studies
have suggested conservation of CT positioning through evolution \cite{Tanabe2002}
and shown tissue specificity in chromosome pairing \cite{Parada2004b}.

It's known that for some but not all genes, positioning at the nuclear
periphery correlates with reduced gene expression \cite{Kosak2002,Dietzel2004,Zink2004a}.
To investigate whether transcriptional activity is the cause or effect
of this positioning effect, two groups published studies which artificially
tethered genomic regions to the inner nuclear membrane using\emph{
}Lac operators \emph{(lacO)} \cite{Finlan2008,Reddy2008}. Both groups
observed a decrease in the transcriptional activity of the regions
when tethered, an effect that was ablated when cells were treated
with trichostatin A (\nomenclature{TSA}{Trichostatin A}TSA) to inhibit
class I and class II histone deacetylases \cite{Finlan2008}. Whilst
these studies suggest that it is nuclear positioning that leads to
transcriptional effects, a study of CT positioning by Croft \emph{et
al.} showed that inhibition of transcription causes a reversible change
in CT position \cite{Croft1999}. Large scale rearrangements of CTs
have also been observed during cell differentiation \cite{Stadler2004}.
Interestingly, derivative chromosomes that result from balanced translocations
affect the organisation of CTs within the nucleus \cite{Harewood2010},
raising the possibility that the global changes in gene expression
observed after oncogenic translocations could be in part due to changes
in genome organisation \cite{Harewood2010}.




\section{Nuclear compartmentalisation}

Cajal bodies, speckles, nucleosomes


\section{Transcription factories}

One nuclear subcompartment which has come to light within the past
twenty years is the \emph{transcription factory}, foci of hyper-phosphorylated
RNA Polymerase II spread throughout the nucleus. The majority of geneic
transcription appears to take place at transcription factories \cite{Jackson1993a,Osborne2004,Eskiw2008a},
challenging the classical model of transcription found in many text
books.

The term 'transcription factories' was coined by Jackson \emph{et
al.} in 1993. Fluorescence microscopy was used to label the incorporation
of bromouridine triphosphate (\nomenclature{BrUTP}{Bromouridine triphosphate}BrUTP)
into nascent RNA; discrete foci of nascent transcription could be
seen within the nucleus which did not form in the presence of the
RNA Polymerase II inhibitor \textgreek{a}-amanitin \cite{Jackson1993a}.
Further studies showed that these foci contain RNA Polymerase II along
with many other components required for transcription \cite{Iborra1996,Grande1997}.
An ultrastructural study by Eskiw \emph{et al.} used correlative microscopy
with both electron spectroscopic imaging (\nomenclature{ESI}{Electron spectroscopic imaging}ESI)
and fluorescence microscopy to study nuclei sections. ESI can distinguish
nitrogen and phosphorous atoms without labelling, and fluorescence
light microscopy can visualise transcription through labelling BrUTP
in nascent transcripts. The authors found that nascent RNA is almost
always associated with the surface of large nitrogen-rich protein
structures with a diameter of \textasciitilde{}87 nm, comparable in
size to that of a transcription factory \cite{Eskiw2008a}.

\begin{wrapfigure}{o}{0.35\columnwidth}%
\begin{centering}
\includegraphics[width=0.3\columnwidth]{\string"figures/chapter 1/transcription factories/transcription-factories\string".pdf}
\par\end{centering}

\caption[Nascent RNA and transcription factories.]{\label{fig:transcription-factories} \textbf{Nascent RNA and transcription
factories.} (A) Transcription foci in HeLa cells, visualised with
labelled Br-UTP in 100 nm cryosections. Nascent RNA (green) is concentrated
in punctate foci. (B) Model for a nucleolar factory, showing a transcript
with multiple polymerases generating a crescent shaped foci. (C) Model
for a nucleoplasmic factory. Multiple transcribed regions each with
a single polymerase generate a smaller cloud of nascent transcripts.
Taken from Cook \emph{et al.}, Science (2009) \cite{Cook1999}.}
\end{wrapfigure}%


The stability of nascent transcripts was shown as far back as 1981
by Jackson and colleagues \cite{Jackson1981} and the visualisation
of transcription factories has demanded a new model for the action
of RNA Polymerase II \cite{Cook1999}. The revised model proposes
that instead of RNA Polymerase II freely diffusing to active genes
and tracking along the gene body, it is the genes that are recruited
to transcription factories and the genes that track through a stationary
polymerase. Such a model provides a better explanation for the mechanics
of transcription - clustering of transcriptional activity may enable
the cell to conduct transcription in a much more efficient manner;
HeLa nuclei have a 1 \textmu{}M concentration of active RNA Polymerase
II, whereas the local concentration within transcription factories
is closer to 1 mM \cite{Carter2008}. Additionally, a polymerase enzyme
moving along a gene would rotate with the helix of the DNA wrapping
the nascent transcript around the template. Genes pulled through static
transcription factories would extrude their RNA transcripts into the
nucleoplasm \cite{Iborra1996,Cook1999}, creating topological loops
within the template DNA instead which may be removed through the activity
of topoisomerases.

Jackson \emph{et al.} went on to work on a quantitative analysis of
transcription factories in HeLa cells calculating the number of active
RNA polymerases, the number of transcription sites and the number
of polymerases associated with each transcriptional unit \cite{Jackson1998}.
They showed that each HeLa cell nucleus contain \textasciitilde{}
2400 transcription factories, each with approximately 30 active RNA-Polymerase
II complexes \cite{Jackson1998}. Importantly, this study showed that
there are more transcribing units than there are foci of transcription,
suggesting that genes must colocalise to transcribe. The number of
transcription factories varies a great deal amongst cell types, but
the observation that genes colocalise within transcription factories
has been confirmed by a number of different techniques; Osborne \emph{et
al.} showed that genes situated within transcription factories are
actively transcribed, whereas those outside are not \cite{Osborne2004}.
They went on to demonstrate that transcription is a discontinuous
process, with the frequency of nascent RNA transcription foci related
to cellular mRNA concentrations, suggesting that transcription occurs
in bursts. Multiple genes both in \emph{cis} and in \emph{trans} were
seen to dynamically colocalise in transcription factories, supporting
predictions that genes must share transcription factories \cite{Osborne2004,Jackson1998}.


\subsection{Specialised transcription factories}

\cite{Schoenfelder2010}


\section{Looping}


\subsection{Chromatin hubs}

Hbb, Igh and other classical looping examples.


\subsection{Long range interactions}

Sonic the hedgehog, other long range \emph{cis} interactions? Any
evidence of \emph{trans} stuff? Cam's Myc and Igh FISH paper.


\section{Domains}

Hi-C domains. Interactions within domains. Talk about how transcription
depends on context. Spreading of epigenetic marks.


\section{What drives nuclear organisation?}


\subsection{Actin and myosin}


\subsection{Transcription}

Active RNA Polymerases: Mobile or Immobile Molecular Machines? \cite{Papantonis2010}


\subsection{CTCF and cohesin}

Jackson, McCreedy and Cook showed in 1981 that nascent RNA transcripts
labelled with {[}\textsuperscript{3}H{]} uridine remained within
the nucleus when loops of DNA were removed using a nuclease, suggesting
that transcription occurs close to points of chromatin attachment
\cite{Jackson1981}.

Igh forming rosettes for v region selection - http://www.ncbi.nlm.nih.gov/pmc/articles/PMC3070187/


\subsection{Transcriptional hubs}

Talk about latest theories about housekeeping genes organising everything
else. How this ties in with chromatin domains.


\subsection{The self-organisation model}

Cook 2002, Iborra and Cook 2002.


\section{The annotated genome}

Talking about the use of a systems biology approach to understand
the linear genomic information of the genome.


\section{Chromosomal translocations}


\subsection{\emph{BCR} and \emph{ABL1}}


\subsection{Mixed lineage leukaemia}


\subsection{Formation of chromosomal translocations}

Basics of HR, SSA and NHEJ. Contact first / breakage first models.


\subsubsection{Double strand breaks and religation}


\subsubsection{Contact first model}


\subsubsection{Chromosome Territories}

Branco and Pombo, Plos Biol 2006

\cite{Parada2004b} - tissue specific CT pairing in mouse, correlates
with tissue translocation incidence


\subsubsection{Transcription Factories}

Cam's Myc and Igh RNA-FISH paper.


\section{Hypothesis and aim of thesis}

Talk about the why of translocations - specificity and frequency of
certain translocations means that partner selection can't be completely
random. Hypothesis and aims: that specific gene pairs are predisposed
to chromosomal translocations because they are co-localising at transcription
factories. Aims of the project are to investigate the genome-wide
association networks of specific oncogenes and see if gene pair co-localisation
frequency matches up with gene-pair translocation frequency.

%\begin{comment}
%\bibliographystyle{plain}
%\bibliography{/Users/phil/Dropbox/Work/BibTeX/THESIS}
%\end{comment}

\end{document}
